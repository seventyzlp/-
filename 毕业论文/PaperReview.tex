\documentclass[]{ctexart}
\title{基于注视点的渲染优化方法综述}
\author{张乐平}


\begin{document}
	\maketitle
	\section{研究背景和意义}
	随着三维游戏场景的复杂化与显示器硬件规格的不断提升,GPU单位时间内需要处理的渲染图元数量和像素数量不断增加。对于图元的着色已经成为局限实时渲染速度的关键因素,在不改变计算机硬件的同时,减少必要处理的图元数量,的确能够提高渲染的效率。
	
	注视点渲染(Foveated Rendering)是指一种利用视觉等效性,基于感知图形算法,降低人眼非注视区域的渲染精度,提高GPU处理图片的效率的方法。
	近些年来,基于注视点的渲染优化方法在较大FOV的显示场景下有广阔的应用,如VR头戴显示器和户外投影设备。
	
	然而,对于平面显示器,这个用户数量多,应用范围广的使用场景,并没有直接使用注视点渲染优化应用帧数,而更多的集中在
	优化HUD显示方式和随着用户目光聚焦位置而扩展用户视野的功能上。对于更大视场角的平面显示器注视点渲染的应用研究,有继续研究的价值和意义。
	
	\section{研究发展现状}
	
	
		\subsection{基于感知的渲染}
		视觉等效性(Visual equivalence)是指人眼在看到传递了相同视觉映像的图片时,会觉得他们时相同的,
		即使这两张图片实际上并不相同。Ganesh Ramanarayanan等人\cite{Ramanarayanan2007VisualET}总结,视觉感知被分为形状感知,材质感知和光照感知。
		并且证明了适当程度的修改光照贴图,并不会影响最终渲染得出图形所给出的视觉映像。
		
		利用这一点,来优化渲染流程,便产生了感知图形算法(Perceptual graphics algorithms)利用人类视觉系统 (HVS) 的特点,只渲染我们实际感知到的内容,从而减少了着色计算时间\cite{Guenter2012Foveated3G}。
		K. Vaidyanathan等人\cite{Vaidyanathan2014CoarsePS}提出了CPS渲染优化方法,在人眼看不清楚的地方适应性降低渲染精度
		来减少GPU处理时间。
		
		更进一步来说,我们的视线垂直方向为 135°,水平方向为 160°,但只能感知中央 5°范围内的细节。这一小部分视野投射到被称为眼窝的视网膜区域,那里密布着色锥受体\cite{Guenter2012Foveated3G}\cite{10.1145/355017.355033}。
		H. Strasburger等人对人眼注视点边缘的感知进行了建模\cite{Strasburger2011PeripheralVA},
		最小可辨角度(视觉敏锐度的倒数)随着偏心率的增加而大致呈线性增长。
		利用这一点,通过跟踪注视点的变化,并根据偏心率调整图像分辨率和几何细节级别(LOD),
		我们可以省略未感知到的细节,绘制出更少的像素和三角形,从而减少渲染所需要的时间。
		
		基于注视点的渲染可以根据LOD的处理形式被分为两种:基于图形的渲染和基于模型的渲染。前一种针对屏幕空间中的像素进行操作,
		后一种对模型空间的顶点进行操作。\cite{1020335127.nh}
		
		从屏幕空间的角度上说,C. Loschky等人基于注视点跟随显示器,研究影响感知效果和渲染性能的相关参数。他们研究的两个关键内容是注视点渲染的耗时,以及用户对周边退化图像的感知能力\cite{10.1145/355017.355032}。 提出,要想让图像质量的变化不被用户察觉,就必须在眼动结束后的5ms内完成场景渲染。
		
		从模型空间的角度上说,B. Guenter使用曲面细分方式来构建各向异性的网格,以提供高质量的注视点渲染结果,通过相机与物体中心之间的距离来调整不同区域的曲面细分水平\cite{Guenter2012Foveated3G}。

		目前,对于注视点渲染的前沿应用场景集中在VR头戴显示器上,有虚拟展厅Zerolight,VR游戏Pavlov VR,
		和Autodesk VRED模型编辑器等应用产品。 Nvida使用了可变速率渲染(Variable Rate Shading)\cite{VRS},和可变速率超级采样 
		能够做到在无额外编码的情况下,使用集成在显卡内的驱动预设。
		
		\subsection{眼动感知装置}
		
		眼动传感器,指的是通过捕获眼球转动的幅度,计算出注视点在屏幕上的位移\cite{825279}。在早期探索时期,Javal通过直接观察使用者的眼睛,并且记录运动轨迹,以及记录眼球运动所产生的不同生物电信号,和Robinson\cite{4322822}提出的在人眼放置线圈,通过对感应电压的侦测,来记录眼球运动的轨迹。
		
		现代眼球跟踪仪最常用的技术是瞳孔中心角膜反射(PCCR)。它使用近红外相机或其他光学传感器来跟踪注视方向。在这种方法中,近红外线照射到瞳孔(即眼睛的中心),在角膜上产生可见的反射,由摄像头进行跟踪\cite{8122153}。
		
		现代眼动仪技术的发展大致从2000年开始,早期的眼动仪(NAC Eye Mark eye tracker)\cite{10.1145/91385.91449}体积硕大,用户
		在交互的时候头部的旋转收到相对较大的制约。
		
		在2001年Tobii成立后,逐渐推动眼动仪小型化的发展。现代眼动传感器大多为挂载在显示器下方的多个红外摄像机阵列组成,并且有
		不同的捕获帧率可以选择,如Tobii Pro Spectrum(1200hz)、Tobii Pro Fusion(250hz)和Tobii Pro Spark(60hz)。在面对大显示器,或是现实移动场景中
		对于眼动追踪的需求,头戴式眼动传感器现如今已经和普通的平光眼睛相同,用户的体验并不会受到太大的影响,例如Tobii Pro Glasses 3。
		
		国内眼动仪技术的发展起步较晚,早期国产眼动仪有田弘杰、孙复川等人采用基于红外传感器的光学方案\cite{YXNX200402010}\cite{SWGC198904009},
		并且也基于此进行了阅读研究的应用。如今,国产眼动仪在商业化、产品化的发展迅速,七鑫易维、EyeSo等企业向单位与个人出售用于
		科研数据采集的眼动传感器。
		
		近年来用户对VR设备画面要求逐渐增加,VR HMD上的眼动传感器为使用注视点渲染优化场景帧数奠定了基础。VR头显的眼动传感器
		原理同头戴式眼动传感器相同。
		Katsuyoshi等人指出,用户在使用VR时,没有外部漫射的环境光源干扰,并且能够固定人眼与传感器之间的空间位置,在一定程度上增加眼动追踪的精确度\cite{8798030}。
		如今的中高端VR头显都配备了眼动传感器,例如HTC Vive pro eye(2019),Pico Neo 2 Eye(2020),HP Reverb G2(2021),
		Sony PSVR 2(2022)等。
		
	\section{评述}
		从如上国内外学者的研究成果而言,对于使用注视点技术优化渲染流程,提高GPU工作效率已经有了较为完整的体系结构,并且在
		VR场景中的优化渲染已经有相对成熟的应用。但是在平面显示器,尤其是超过27寸的大型显示器上,挂载的眼动感知装置就不能支持。
		并且,对于平面显示器渲染优化上的应用目前相对欠缺,有待开发平面显示器的渲染优化和基于眼动的交互控件的应用。
		
	\bibliographystyle{unsrt}
	\bibliography{ref}
\end{document}










