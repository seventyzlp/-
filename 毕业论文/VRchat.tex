\documentclass[]{article}

\title{Exploring the Metaverse’s Concept of “Connection” Through VRChat}
\author{Zhang Leping}
\date{}

\begin{document}
	\maketitle
	As I write this article, I have played VRChat for 60 hours and believe I have gained some insights that may be of interest to readers. The future should be empowered by "connection," rather than focusing on the commercial or conceptual value of virtual assets. The development of internet culture has led to hot topics emerging and shifting rapidly, largely driven by the spread of short videos. Everyone is consuming the entertainment value of what they encounter, creating their own information bubbles. The concept of the metaverse, which extends beyond the “global village,” should focus on connection as its core—whether by bridging physical distances to enable cloud-based work or psychological distances to foster deeper internet integration.
	
	\section{Virtual Spaces Create Opportunities for Connection}
	VRChat, as a VR social game, allows players to choose an avatar as their online persona. If they don't want to speak, they can simply type, greatly lowering the barriers to communication. After all, you are just an anonymous entity on the internet, using a pseudonym and an avatar, without worrying that your words or actions might have irreparable consequences. Even if you don’t get along with someone, it doesn't matter—VRChat's large player base ensures that you will find someone you resonate with. You can simply block those with whom you don't connect, avoiding the complexities of real-life social interactions. VRChat encourages players to express themselves freely, creating opportunities for connection in the virtual (metaverse) space.
	
	As a result, VRChat has become a gathering place for people who are socially anxious, lonely, depressed, or feel misunderstood by mainstream values. They, myself included, use VRChat to establish cyber-social connections, though these bonds are often fragile. Through the game's inclusive and wide-reaching social environment, people express their desires, political views, or simply vent their frustrations with the world. VRChat serves as a warm and inclusive haven, where communities connect, offering a space for those yearning to express themselves.
	
	However, the metaverse also harbors dangers. Some socially adept players might become predators, weaving clever internet scams to fulfill their desires or make money. There are also cyber vigilantes who seek justice. More commonly, tensions arise between groups led by dominant players, who are often cliquish, making it difficult for new or “wild” players to integrate.
	
	Yet, whether for noble or malicious purposes, forming relationships in virtual spaces is often easier than in the real world.
	
	\section{Cyber-Socializing Leads to Greater Real-World Emptiness}
	When players forge cyber friendships and turn VRChat into their personal paradise, taking off their VR headsets makes the real world seem crueler by comparison. Everything external to the soul comes rushing back, and after leaving the cyber world, one’s sense of self, shaped by real-world circumstances, reattaches itself.
	
	This sense of emptiness arises from two factors. First, leaving the warm, safe environment of VRChat requires greater effort and cost to rebuild one's real-world life. Second, the bonds formed in cyber spaces are often fragile; once you leave the game, these friendships may vanish, and the joy experienced in VR may not translate into real life. Furthermore, immersing oneself in VRChat requires a significant amount of time, leaving little energy to build a life in the real world, let alone make friends.
	
	Clearly, the fast-paced social habits cultivated by the metaverse force players to invest more money and time into VRChat, blurring the lines between reality and fantasy. People seek refuge in an ever-expanding bubble, experiencing temporary comfort, though the day will inevitably come when the dream bursts. The more you immerse yourself in VRChat’s social environment, the more you realize how small and pitiful you feel once you take off your headset and trackers. Addiction to the virtual world inevitably leads to greater real-world emptiness. Is the sense of connection provided by the metaverse truly what we want?
	
	\section{The Incompleteness of VRChat as a Game}
	VRChat's success lies in its separation from the real world—it is, after all, just a game. One major reason why VRChat cannot become something like the "Oasis" from Ready Player One is the lack of a strong link between the game and reality.
	
	In Ready Player One, the game's currency could be used to buy real-world items, bridging the gap between the virtual and the real. In this way, real-world ethics and social norms were integrated into the metaverse, blurring the boundaries between reality and the virtual world. As such, the emptiness caused by cyber-socializing would disappear.
	
	However, I believe that the internet should remain separate from real society, serving as fertile ground cultivated by internet natives. If we were to build a bridge to reality, while the virtual bubble might not burst, its inner essence would lose its sweetness. Real-life failures would drive those seeking escape back into isolation, just as Native Americans once embarked on another migration. Therefore, perhaps it is better to let VRChat remain a game, one in which player interactions do not extend to the physical world, preserving the purity of the internet.
	
	Yet, the intrusion of reality into the virtual is inevitable. Online scams, as mentioned above, are just one typical example. Other paid services, such as "companionship" services and entertainment projects like murder mystery games, also demonstrate the ways in which real-world capital is attempting to extract profits from the virtual realm. This method of extracting "meta-profits" will likely become more prevalent in the future, and today’s so-called "conventions" are just the initial attempts by real-world capital to stake a claim in the vast virtual market.
	
	In short, VRChat is at the edge of both the gaming and real-world spectrums.
	
	\section{Does the "Oasis" Really Exist?}
	In Ready Player One, the "Oasis" is portrayed as a beautiful, sacred place, with little mention of its darker aspects. However, the fully integrated metaverse is not as idyllic as we might imagine. The Oasis will undoubtedly exist in the near future, but it won’t be the utopia we envision.
	
	The Oasis will be a model world, under strict internet regulation, and it will certainly not be built by an individual. For example, minors might only be allowed to play for an hour and a half on weekends.
	
	However, remote work and internet collaboration will develop significantly, and productivity will greatly increase. Many real-world jobs will transform into virtual ones, allowing the metaverse to fully reflect and encapsulate real life.
	
	A virtual world that mirrors reality is not the utopia we hope for, so in a sense, the "Oasis" does not truly exist.
	
	\section{Overcoming the Barriers of a Fragmented Internet Through Connection}
	The only way to break down the growing emotional barriers between people is through connection. The internet has turned the world into a global village, removing geographical barriers, while the metaverse will allow internet users to confront each other directly, breaking down communication barriers. However, creating new social bonds after severing old ones to achieve a sense of self-identity is extremely difficult. Many people just want a cyber confessional where they can freely express themselves—they don’t take their cyber-social connections seriously.
	
	For established groups, security concerns prevent them from accepting new players of unknown origin. As a result, new users entering the "metaverse" often find it difficult to establish their own bonds, while also needing to evaluate whether their communication partners are safe and trustworthy.
	
	While connection is the right choice, achieving it in reality is very challenging. Even in the internet age, connection is difficult because the people involved remain the same. But the internet has given everyone the chance to express themselves, and the metaverse has brought these voices together, making them heard. This breaks down the previous barriers of the internet, creating the conditions for human connection. As personal networks grow, the world becomes more accessible, continuing to fracture the closed information environments.
	
	Breaking down information bubbles through connection is inevitable, and creating an inclusive social environment will ultimately connect us all.
\end{document}