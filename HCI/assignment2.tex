\documentclass{article}
\title{Lo-Fi evaluation feedback}
\author{ZHANG LEPING}
\date{\today}
\begin{document}
	\maketitle
	
	\begin{abstract}
		This is a usability evaluation report which is the feedback to my collaborator.
		The review is mainly based on the "8 Golden Rule" in user experience design.
		
		\noindent\textbf{Designer: }KONRAD ELISA \qquad
		\textbf{Email: }N2301643F@e.ntu.edu.sg
		
		\noindent\textbf{Reviewer: }ZHANG LEPING \qquad
		\textbf{Email: }N2301172D@e.ntu.edu.sg
		
		
	\end{abstract}
	
	\section{Strive for consistency}
	
	To maintain the consistency in UI design, the control bar in the lower part of the screen which plays very important role in the app has same appearance. And all the icon or buttons
	are designed with round corners, which represent a consistency in the art design.
	
	From another point of view, the consistency in human interaction with other map applications is also very important. In the navigation page, designer put the "next crossing" message on top of the screen which is similar to Google Map. And many common icons such as magnifiers represent "search" and gear represent "system configure" are having the consistent meaning in almost all navigation apps.
	
	\textbf{suggestions}
	
	To improve the rule of consistency, I think to put more image in the First login and Main page of the middle screen like the other pages is a good way to present information.
	 
	\section{Cater to universal usability}
	
	Universal usability means the interface need to meet the needs of the widest possible range of users.
	With the input message about age, walking speed and etc, the application will change the recommendation
	algorithm in planning route to the destination. 
	
	Considering users with different languages, it is allowed to change the interface language in the main page. There is a help icon in every page to help novice user to come across not only language barrier, but also give practical hints and step-by-step tutorial about the application. 
	And a support button allow users to chat with AI to meet some special requests.
	
	\textbf{suggestions}
	
	To improve the rule of usability, putting a help and language change button in the First login page
	seems a good choice. Because the novice user who can't speak English will find it very hard to input his
	personal information, which may lead to some misunderstandings.
	
	\section{Offer informative feedback}
	
	The rule about feedback means the system should provide instant, meaningful and practical information during the hole interaction procedure. In the navigation page, user's direction and position will be
	demonstrated and update frequently. So users in the evaluation sequence will feel the system more controllable.Further more, the AR mode will represent real-time position change and navigation.
	
	The Internet connected AI agent is able to start a conversation with users, providing practical information or responding to questions in time. And the application will offer quick reaction if the user are reaching his/her destination and return to the main page automatically.
	
		\textbf{suggestions}
		
	However, the user with AR mode turning on, will concentrate all his/her attention to the camera image. It seem dangerous while crossing roads with heavy traffic. May add more information about the environment around users will help to reduce these risk. 
	
	\section{Design dialogs to yield closure}
	This application divided users action sequences into several steps, as Ben Shneiderman said. The user will first input his/her personal information to build up the recommendation database. After that users will start route planning choice a way start the navigation. After reaching the final point, the sequence of action end, system initialized and waiting for another loop.
	
	\section{Permit easy reversal of actions}
	
	This rule means the system need to allow users to reveal choices decided. In this prototype, there is a return button on top of the screen for users to exit the current page. And during the navigation, it is also reachable for users to edit the current route, comparing with different indoor or outdoor transpositions.
	
		\textbf{suggestions}
	
	But the sketch do not show how to exit AI mode. If the AI page is attached to the current page, clicking the exit button on top of the current page will cause some misunderstandings. Or we need to say "goodbye" to close the Ai agent during the chat?
	
	\section{Support internal locus of control}
	
	This rule means systems should not ignore user's operations and react to that operation. The real-time reaction on the map position based on the user will form the feel of control. Besides, the application will estimating how much time/meters left to reach destination and show it on top of the screen.
	
		\textbf{suggestions}
	
	To strengthen the feel of control, I think when app fatal crushed, may set a button or some ways to reset the entire application. 
	
	\section{Reduce short term memory}
	
	Short term memory means the item need user to remember for little minutes, such as OTP for form filling.
	With auto-filling and context sensing in typing to search for some places to go, user will not be asked to remember a entire name of the destination which will help reduce short term memory.
	
	All the route hint will follow a step-by-step guidance, so the user can just follow the instructions that appear on the screen. And the procedure of using this app is simple to understand, which may not increase memory load. Besides, the interface design with a lot widely-used icon. For example, the search bar and the microphone button. These icons shared a common meaning with other interfaces.
	
		\textbf{suggestions}
	
	To release users' memory load, simplify the first login page process may help novice users to get started.
	
	\section{Prevent errors}
	
	The first error preventing method in this lo-fi prototype, is to limit user's choices. The select list box in First login page and main page will restrict user to select available options. 
	
	If users are having trouble to deal with the error/issue occurred, the AI agent will able to help user.
	Customer service will step in to help resolve the issue, if AI failed. That is the most powerful way to prevent errors.
	
		\textbf{suggestions}
	
	To prevent time conflict errors, providing the available time of the destination to users and warning if it will close when reaching.
	
	\section{Additional Comments}
	
	The sketch work is well done to represent the major user experience, especially the indoor navigation.
	Considering the representation of travel crossing sky-walks or some complex public buildings in 2D map,
	may be a good choice.
	
	And how do users locate or calibrate their location inside the building using the app is a practical feature.
	
\end{document}
